\label{section:vertex-area-chapter}
This section shows alternative methods to extend the idea of flat shading from triangles to vertices. The idea of flat shading is to draw all the pixels of a triangle with the same colour. The extension of this approach is to split the surface of the triangle mesh likewise into regions around vertices and draw all pixels in these regions with the same colour (Fig. \ref{fig:vertex-area}), thus visualizing data given at the vertices of the mesh in a piecewise constant, not necessarily continuous way, resembling the classical triangle flat shading. The aforementioned regions can easily be defined using barycentric coordinates and a simple GPU fragment program (Fig. \ref{fig:max-diagram}) can be used for each pixel to find out to which region it belongs and which colour it should be painted with.

%%%%%%%%%%%%%


\begin{figure}[!h]
    \centering
    \minipage[b]{.5\linewidth}
    \centering
    \includegraphics[scale=0.15]{images/max.png}
    \caption{Max diagram}\label{fig:max-diagram}
    \endminipage\hfill
    \minipage[b]{.5\linewidth}
    \centering
    \includegraphics[scale=0.15]{images/vertex-area.png}
    \caption{Region around a vertex}\label{fig:vertex-area}
    \endminipage
\end{figure}

%%%%%%%%%%%%%

\subsection{Max diagram - Vertex based area} \label{section:max-diagram}
Passing barycentric coordinates to the \textit{fragment shader} will clearly demonstrate that we can get results different from the classic color interpolation.\cite{WEBSITE:redbloggames}
%----------
There are various approaches to color interpolation focusing on the distance from vertices. For each point in a triangle, we can easily determine its closest vertex, which we use as a cue for coloring.
Another approach, different from the above, can be defined as coloring vertex areas based on the maximum barycentric coordinate.
The color is given by the region closest to a vertex (Fig. \ref{fig:max-diagram}, Pseudocode \ref{appendix:max-diagram}).

%%%%%%%%%%%%%

\subsection{Vertex Flat Shading} \label{section:extend-flat-shading-lighting}
An extension of \textit{flat shading} would be to have each vertex area to be in one constant color. This color can be taken using the normal at the vertex and the vertex position.
The color will then be computed as in \textit{Gouraud shading}.
The idea is to compute the color per vertex but instead of linearly interpolating it in each triangle (as \textit{Gouraud shading} does) we color regions around a vertex with that constant color (using the GPU fragment program: max diagram \ref{section:max-diagram}).
To implement this approach, the barycentric coordinates, the vertex color, the normal at the vertex and the lighting calculations must be passed to the \textit{fragment shader}.
We want to avoid the automatic interpolation of colors. In order to return the resulting color using the \textit{max diagram}, we have used a \textit{Geometry shader} that has access to all three vertex colors in \textit{fragment shader}. (Pseudocodes: \ref{appendix:vs-flat-shading-lighting}, \ref{appendix:gs-flat-shading-lighting}, \ref{appendix:fs-flat-shading-lighting})

\begin{figure}[!h]
    \centering
    \minipage[b]{.3\linewidth}
    \centering
    \includegraphics[scale=0.3]{images/armadillo-extendfs.png}
    \endminipage\hfill
    \minipage[b]{.3\linewidth}
    \centering
    \includegraphics[scale=0.3]{images/armadillo-extendfs-1.png}
    \endminipage\hfill
    \minipage[b]{.3\linewidth}
    \centering
    \includegraphics[scale=0.3]{images/armadillo-extendfs-2.png}
    \endminipage
    \caption{Vertex flat shading.}
\end{figure}


\subsection{Comparison between triangle flat shading, triangle Gouraud shading and vertex flat shading}
The standard approaches: \textit{triangle flat shading} and \textit{triangle Gouraud shading} are then compared with the new technique \textit{vertex flat shading}. In Fig. \ref{fig:comparison-icosahedron} we can see that the icosahedron where we have applied the vertex flat shading shader seems to be a good compromise since it preserves the original geometry and avoids creating triangle-like artifacts in the final result. Vertex regions look more realistic and less noisy than triangle regions. Moreover, Gouraud shading is prone to smoothing and loosing many details in some meshes.
\begin{figure}[!h]
    \centering
    \minipage[b]{.3\linewidth}
    \centering
    \includegraphics[scale=0.5]{images/flatshading.png}
    \label{fig:flat-shading-triangle}
    \endminipage\hfill
    \minipage[b]{.3\linewidth}
    \centering
    \includegraphics[scale=0.5]{images/gouraudshading.png}
    \label{fig:gouraud-shading}
    \endminipage\hfill
    \minipage[b]{.3\linewidth}
    \centering
    \includegraphics[scale=0.5]{images/extentflatshading.png}\label{fig:flat-shading-vertex}
    \endminipage
    \caption{Comparison between: triangle flat shading, triangle Gouraud shading and vertex flat shading.}
    \label{fig:comparison-icosahedron}
\end{figure}

\begin{figure}[!h]
    \centering
    \minipage[b]{.33\linewidth}
    \centering
    \includegraphics[scale=0.6]{images/genus-fs.png}
    \endminipage\hfill
    \centering
    \minipage[b]{.33\linewidth}
    \centering
    \includegraphics[scale=0.6]{images/genus-gs.png}
    \endminipage\hfill
    \minipage[b]{.33\linewidth}
    \centering
    \includegraphics[scale=0.6]{images/genus-efs.png}
    \endminipage\hfill
    \minipage[b]{.33\linewidth}
    \centering
    \includegraphics[scale=0.35]{images/eight-fs.png}
    \endminipage\hfill
    \centering
    \minipage[b]{.33\linewidth}
    \centering
    \includegraphics[scale=0.35]{images/eight-gs.png}
    \endminipage\hfill
    \minipage[b]{.33\linewidth}
    \centering
    \includegraphics[scale=0.35]{images/eight-efs.png}
    \endminipage\hfill
    \centering
    \minipage[b]{.33\linewidth}
    \centering
    \includegraphics[scale=0.6]{images/armadillo-fs.png}
    \endminipage\hfill
    \centering
    \minipage[b]{.33\linewidth}
    \centering
    \includegraphics[scale=0.6]{images/armadillo-gs.png}
    \endminipage\hfill
    \minipage[b]{.33\linewidth}
    \centering
    \includegraphics[scale=0.6]{images/armadillo-efs.png}
    \endminipage\hfill
    \minipage[b]{.33\linewidth}
    \centering
    \includegraphics[scale=0.39]{images/horse-fs.png}
    \endminipage\hfill
    \centering
    \minipage[b]{.33\linewidth}
    \centering
    \includegraphics[scale=0.39]{images/horse-gs.png}
    \endminipage\hfill
    \minipage[b]{.33\linewidth}
    \centering
    \includegraphics[scale=0.39]{images/horse-efs.png}
    \endminipage\hfill
    \caption{On the left: Triangle flat shading. On the center: Triangle Gouraud shading. On the right: Vertex flat shading.}
    \label{fig:comparison-gc-gci}
\end{figure}

%%%
\subsection{Gaussian curvature}
\label{section:vertex-area-gaussian-curvature}
Another interesting alternative data visualization technique is to compute the \textit{Gaussian curvature} per vertex. That can be done summing up, for each vertex, angles at this vertex with adjacent triangles and then subtracting this value from $2\pi$.
Having obtained this value, called \textit{angle defect} (Fig. \ref{fig:gc-angle}), we map it linearly to a color range.
The resulting color will be the vertex flat shading visualisation of \textit{Gaussian curvature} (See Section \ref{section:localaveraging} and \ref{section:gaussian-curvature-intro}).
$$K(V) = (2\pi - \sum_j \theta_j)/\mathcal{A}_{Mixed}$$
%%
\begin{figure}[!h]
    \centering
    \scalebox{0.65}{\begin{tikzpicture}
        \coordinate (J) at (3.8,3.6);
        \node[anchor=south west,inner sep=0] at (0,0) {\includegraphics[scale=0.25]{images/vertex-area.png}};
        \draw (J) node [below left] {$j$};
        \filldraw (3.5,2.7) circle (2pt);
        \begin{scope}[line width=0.4mm, line cap=round]
            \draw (3.9,2.2) arc (295:360:0.7cm) node[near start,right] {$\theta_j$};
        \end{scope}
    \end{tikzpicture}}
    \caption{On the left: angle defect is denoted with $\theta_j$.}\label{fig:gc-angle}
\end{figure}
%%

%%%%%%%%

\subsection{Constant Gaussian curvature per vertex}
\textit{Constant Gaussian curvature per vertex} returns a constant color around each vertex (Fig. \ref{fig:gc-angle}, Pseudocode \ref{appendix:vs-gaussiancurvature}). Calculating the \textit{Gaussian curvature} per vertex, this value is mapped into a color range to get the curvature color (see Fig. \ref{fig:color-range-curvature}). This process is made separately for each vertex of the triangle and consequently, using the technique of max-diagram explained above (See section \ref{section:max-diagram}), the final resulting constant color is returned.
\begin{figure}[!h]
    \centering
    \includegraphics[scale=1.2]{images/gradient-curvature.png}
    \caption{Color bar showing respective colors for negative, flat and positive curvatures. Negative curvatures are colored in red, flat curvatures in green and positive curvatures in blue.} \label{fig:color-range-curvature}
\end{figure}


\begin{figure}[!h]
    \centering
    \minipage[b]{.5\linewidth}
    \centering
    \includegraphics[scale=1.0]{images/gc-armadillo-top.png}
    \endminipage\hfill
    \minipage[b]{.5\linewidth}
    \centering
    \includegraphics[scale=0.4]{images/gc-detail-armadillo-top.png}
    \endminipage
    \caption{Constant Gaussian curvature per vertex} \label{fig:gc-detail}
\end{figure}



\subsection{Gouraud Gaussian curvature}
\textit{Gouraud Gaussian curvature} returns an interpolated color per vertex. The idea is to calculate the \textit{Gaussian curvature} as explained above (mapping the color into a color range to get the corresponding color per vertex) but instead of returning the constant color using a max-diagram approach, we just return the interpolation of values obtained for each triangle.

\begin{figure}[!h]
    \centering
    \minipage[b]{.5\linewidth}
    \centering
    \includegraphics[scale=1.0]{images/gci-armadillo-top.png}
    \endminipage\hfill
    \minipage[b]{.5\linewidth}
    \centering
    \includegraphics[scale=0.4]{images/gci-detail-armadillo-top.png}
    \endminipage
    \caption{Gouraud Gaussian curvature} \label{fig:gci-detail}
\end{figure}


\subsection{Evaluation and Comparison between constant Gaussian curvature per vertex and Gouraud Gaussian curvature}
We compare the \textit{constant Gaussian curvature per vertex} (Fig. \ref{fig:gc-detail}) with the \textit{Gouraud Gaussian curvature} (Fig. \ref{fig:gci-detail}).
In Fig. \ref{fig:gc-detail} each vertex area is colored applying the method \textit{max diagram}. Instead, in Fig. \ref{fig:gci-detail} the color is obtained with a linear interpolation.
Visualization of the principal curvatures of the model as colors from blue (highest values of curvature) to red (lower values of curvature) in Fig. \ref{fig:comparison-gc-gci} highlighs the geometry of meshes.
These changes of curvature, positive (blue), flat (green) and negative regions (red), better emphasizes the 3-dimensionality of the model.
\textit{Gouraud Gaussian curvature} is smoother, which results in a loss of small details. This is particularly evident in armadillo's legs mesh. Instead, \textit{constant Gaussian curvature per vertex} generates sharper edges with piecewise-flat regions which slightly degrades the 3-dimensional perception of the model.
On the other hand, \textit{Constant Gaussian curvature per vertex} preserves the details of the given geometry, which can be particularly useful for data visualization purposes.

\begin{figure}[!h]
    \centering
    \minipage[b]{.5\linewidth}
    \centering
    \includegraphics[scale=0.7]{images/gc-genus.png}
    \endminipage\hfill
    \minipage[b]{.5\linewidth}
    \centering
    \includegraphics[scale=0.7]{images/gci-genus.png}
    \endminipage\hfill
    \minipage[b]{.5\linewidth}
    \centering
    \includegraphics[scale=0.75]{images/gc-eight.png}
    \endminipage\hfill
    \minipage[b]{.5\linewidth}
    \centering
    \includegraphics[scale=0.75]{images/gci-eight.png}
    \endminipage\hfill
    \minipage[b]{.5\linewidth}
    \centering
    \includegraphics[scale=0.75]{images/gc-armadillo.png}
    \endminipage\hfill
    \minipage[b]{.5\linewidth}
    \centering
    \includegraphics[scale=0.75]{images/gci-armadillo.png}
    \endminipage\hfill
    \minipage[b]{.5\linewidth}
    \centering
    \includegraphics[scale=0.75]{images/gc-horse.png}
    \endminipage\hfill
    \minipage[b]{.5\linewidth}
    \centering
    \includegraphics[scale=0.75]{images/gci-horse.png}
    \endminipage
    \caption{On the left: Constant Gaussian curvature per vertex. On the right: Gouraud Gaussian curvature.}
    \label{fig:comparison-gc-gci}
\end{figure}

